\section{Recomendaciones}
\label{sec:recomendaciones}


Se recomienda fuertemente la integración de la suite de pruebas automatizadas en un pipeline de integración continua. Actualmente, las pruebas se ejecutan manualmente o mediante el panel de administración, pero vincular la suite a herramientas como GitHub Actions, Jenkins o GitLab CI permitiría la ejecución automática cada vez que se realice un cambio en el código del bot. Esto garantizaría la detección inmediata de regresiones y proporcionaría retroalimentación constante al equipo de desarrollo, reduciendo significativamente el tiempo entre la introducción de un defecto y su identificación.

Relacionado con lo anterior, sería valioso establecer un umbral mínimo de éxito para las pruebas automatizadas como criterio de calidad para despliegues en producción. Por ejemplo, exigir que al menos el noventa por ciento de las pruebas pasen exitosamente antes de autorizar un despliegue. Esto institucionalizaría la automatización como parte fundamental del proceso de aseguramiento de calidad, evitando que los despliegues se realicen con funcionalidad conocidamente defectuosa.

Se recomienda también ampliar la cobertura de pruebas más allá de los flujos exitosos o "happy paths" actualmente implementados. El sistema de pruebas podría extenderse para validar escenarios de error, como manejo de datos inválidos, campos faltantes, valores fuera de rango y situaciones excepcionales. Por ejemplo, probar qué ocurre cuando el usuario intenta crear un cultivo sin proporcionar todos los campos obligatorios, o cuando ingresa un precio negativo para un producto. Estas pruebas de casos límite y manejo de errores fortalecerían significativamente la robustez del bot y mejorarían la experiencia del usuario final al garantizar respuestas apropiadas incluso ante entradas inesperadas.


