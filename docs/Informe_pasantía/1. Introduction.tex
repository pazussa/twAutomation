\clearpage
\pagenumbering{arabic}
\section{Introducción}
El presente informe documenta el trabajo realizado durante la pasantía desarrollada en la empresa 
\textbf{Grandtek}, enfocada en el \textit{desarrollo de soluciones especializadas de software} basadas en 
tecnologías de la información y las comunicaciones (TIC). Grandtek ofrece servicios de mantenimiento y 
soporte informático, así como desarrollo de software especializado para diversos sectores, respaldados por 
un equipo de ingenieros expertos en diseñar soluciones confiables basadas en TIC. 

La pasantía se orientó a la \textbf{automatización de pruebas de conversaciones de WhatsApp} para un bot del 
ámbito agrícola, con el propósito de validar historias de usuario a lo largo de \textit{tres fases del proyecto} de la 
aplicación de Agricultura de Precisión. Este trabajo se articuló en torno al ámbito de la automatización web, 
pruebas end-to-end con la librería Playwright y gestión reproducible de datos de prueba.

El desarrollo de este proyecto se sustenta en principios establecidos de ingeniería de software, automatización de pruebas y sistemas conversacionales. Las pruebas automatizadas son una práctica fundamental en el desarrollo de software moderno. Según Garousi y Mäntylä \cite{garousi2016citations}, las pruebas automatizadas reducen significativamente el tiempo de ejecución de pruebas repetitivas y aumentan la cobertura de código, permitiendo detectar regresiones tempranamente. En el contexto de aplicaciones web, las pruebas end-to-end (E2E) validan el sistema completo desde la perspectiva del usuario final, asegurando que todos los componentes integrados funcionen correctamente en conjunto \cite{leotta2016comparing}. El patrón \textit{Page Object Model} (POM), utilizado en este proyecto, es reconocido como una práctica efectiva para estructurar pruebas automatizadas de interfaces de usuario. Leotta et al. \cite{leotta2013capture} demuestran que POM mejora la mantenibilidad del código de pruebas al encapsular la lógica de interacción con elementos de la interfaz, reduciendo la duplicación y facilitando la adaptación a cambios en la UI. Por su parte, los agentes conversacionales o \textit{chatbots} han experimentado un crecimiento exponencial en diversos dominios. Adamopoulou y Moussiades \cite{adamopoulou2020chatbots} clasifican los chatbots según su arquitectura en: basados en reglas, basados en recuperación de información, y basados en modelos generativos. El bot bajo prueba en este proyecto se enmarca en sistemas híbridos que combinan procesamiento de lenguaje natural con lógica de negocio específica del dominio agrícola.

WhatsApp Web presenta desafíos específicos para la automatización debido a su naturaleza asíncrona, actualizaciones en tiempo real y dependencia de servicios externos. La literatura sobre testing de aplicaciones de mensajería instantánea \cite{amalfitano2012using} destaca la importancia de: (i) esperas inteligentes con estrategias de \textit{polling} para sincronización con el servidor; (ii) manejo robusto de variabilidad en tiempos de respuesta; y (iii) aislamiento de datos entre pruebas para garantizar reproducibilidad. El enfoque implementado en este proyecto, basado en detección de patrones mediante expresiones regulares y sistema de reglas con prioridades, se alinea con las mejores prácticas identificadas en la literatura para testing de interfaces conversacionales no determinísticas \cite{langevin2021testing}.


Los antecedentes que motivaron esta pasantía incluyen la experiencia previa del pasante en 
\textbf{desarrollo web} y \textbf{automatización de pruebas}, lo cual facilitó la adopción de tecnologías como 
TypeScript y Playwright, así como el diseño de \textit{flujos conversacionales automatizados} que interactúan con 
WhatsApp Web. Este punto de partida permitió alinear el trabajo con las necesidades del 
proyecto.

La pasantía se articula con la formación del programa específicamente en el énfasis en \textbf{Telemática}, al integrar patrones de diseño de software, plataformas de 
comunicación en red y pruebas de software para construir soluciones de validación de extremo a extremo. 
Esta experiencia fortaleció competencias técnicas y blandas alineadas con el perfil del estudiante.

El alcance de la pasantía abarcó: (i) la validación de historias de usuario priorizadas en las \textit{tres fases} 
del proyecto; (ii) la definición y materialización de \textbf{datos de prueba versionados} para dominios 
agrícolas (cultivos, fertilizantes, fitosanitarios); (iii) la \textbf{automatización de flujos conversacionales} en 
WhatsApp Web con Playwright; y (iv) la \textbf{estandarización de reportes} de resultados de ejecución.

Entre las principales limitaciones se encuentran: la dependencia de la disponibilidad de \textit{WhatsApp Web} 
y del bot bajo prueba; restricciones de tiempo propias del periodo de práctica que acotan el trabajo a 
las historias priorizadas de Fases 1--3.


\subsection*{Objetivos}
\textbf{Objetivo general.} Desarrollar un conjunto de pruebas automatizadas de WhatsApp que valide las 
funcionalidades de las historias de usuario de las tres fases del proyecto de la app de Agricultura de 
precisión, generando una base para futuras extensiones del proyecto.

\medskip
\noindent\textbf{Objetivos específicos.}
\begin{enumerate}[label=\arabic*)]
  \item \textbf{Modelo de datos de prueba reproducible y seguro.} Diseñar un modelo de datos de prueba 
  para conversaciones, con entregables que incluyan: conjunto de datos versionados (cultivos, fertilizantes, 
  fitosanitarios, etc.), plantillas y selección de dataset por etiqueta/ambiente.
  \item \textbf{Automatización de intercambio de mensajes (Fases 1--3).} Automatizar end-to-end el 
  intercambio de mensajes usuario–bot para las historias de usuario priorizadas de Fase 1, 2 y 3, mediante 
  \textit{steps} y \textit{Page Objects} reutilizables. Entregables: repositorio de steps para todas las fases (enviar 
  mensaje, seleccionar opción, validar respuesta), orquestación por etiquetas de Fase/HU y ejecución 
  paralela local.
  \item \textbf{Generación de reportes estandarizada.} Implementar y estandarizar la generación de reportes 
  post-ejecución con Playwright, consolidando HTML, JSON y evidencia de capturas automáticas. Entregable: 
  script de generación de reportes que produce HTML y reporte de casos exitosos y pasos completados.
\end{enumerate}

















