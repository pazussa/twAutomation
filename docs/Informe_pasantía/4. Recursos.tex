\section{Recursos empleados}
\label{sec:recursos}

A continuación se presenta una descripción detallada de los recursos empleados, organizados por categorías.

\subsection{Recursos de hardware}

El desarrollo del proyecto se realizó en un equipo personal con las siguientes características:

\begin{table}[h]
\centering
\caption{Especificaciones del equipo de desarrollo}
\label{tab:hardware}
\begin{tabular}{|l|l|}
\hline
\textbf{Componente} & \textbf{Especificación} \\
\hline
\textbf{Equipo} & Lenovo IdeaPad 3 14ITL05 \\
\hline
\textbf{Procesador} & Intel Core i5-1135G7 (11th Gen) \\
 & 4 núcleos, 8 hilos \\
 & Frecuencia base: 2.40 GHz \\
 & Frecuencia máxima: 4.20 GHz \\
\hline
\textbf{Memoria RAM} & 4 GB DDR4 \\
\hline
\textbf{Almacenamiento} & SSD NVMe 256 GB \\
 & Espacio utilizado: 55 GB \\
 & Espacio disponible: 166 GB \\
\hline
\textbf{Sistema operativo} & Linux (distribución basada en Ubuntu) \\
\hline
\end{tabular}
\end{table}

\subsection{Recursos de software}

\subsubsection{Entorno de desarrollo y herramientas}

\begin{table}[h]
\centering
\caption{Software y herramientas de desarrollo}
\label{tab:software}
\begin{tabular}{|l|l|l|}
\hline
\textbf{Herramienta} & \textbf{Versión} & \textbf{Propósito} \\
\hline
\textbf{Node.js} & v22.18.0 & Runtime JavaScript/TypeScript \\
\hline
\textbf{npm} & v10.9.3 & Gestor de paquetes \\
\hline
\textbf{TypeScript} & v5.9.2 & Lenguaje de programación tipado \\
\hline
\textbf{Playwright} & v1.55.0 & Framework de automatización de navegador \\
\hline
\textbf{Express.js} & v4.19.2 & Framework web para servidor de administración \\
\hline
\textbf{ts-node} & v10.9.2 & Ejecución de TypeScript sin compilación previa \\
\hline
\textbf{dotenv} & v16.4.5 & Gestión de variables de entorno \\
\hline
\textbf{Git} & -- & Control de versiones \\
\hline
\textbf{VS Code} & -- & Editor de código (IDE) \\
\hline
\textbf{LaTeX} & pdflatex & Generación de documentación técnica \\
\hline
\end{tabular}
\end{table}

\subsubsection{Servicios y plataformas}

\begin{table}[h]
\centering
\caption{Servicios y plataformas utilizadas}
\label{tab:servicios}
\begin{tabular}{|l|l|}
\hline
\textbf{Servicio} & \textbf{Uso} \\
\hline
\textbf{WhatsApp Web} & Interfaz de usuario del bot bajo prueba \\
\hline
\textbf{GitHub} & Repositorio de código y control de versiones \\
\hline
\textbf{Chromium} & Motor de navegador para Playwright \\
 & (incluido en Playwright, modo persistente) \\
\hline
\end{tabular}
\end{table}



\subsubsection{Stack tecnológico}

La selección de tecnologías fue \textbf{apropiada y bien fundamentada}:

\paragraph{Node.js + TypeScript:}
\begin{itemize}
  \item \textbf{Ventajas:} Tipado estático reduce errores, excelente ecosistema de paquetes, compatibilidad nativa con JavaScript del navegador
  \item \textbf{Resultado:} Desarrollo ágil con detección de errores en tiempo de compilación
\end{itemize}

\paragraph{Playwright:}
\begin{itemize}
  \item \textbf{Ventajas:} Contexto persistente integrado (crucial para mantener sesión de WhatsApp), esperas automáticas inteligentes, API moderna y bien documentada, soporte nativo para TypeScript
  \item \textbf{Alternativas consideradas:} Selenium (rechazado por complejidad de configuración), Puppeteer (rechazado por falta de contexto persistente robusto)
  \item \textbf{Resultado:} Ideal para el caso de uso; redujo significativamente el tiempo de desarrollo
\end{itemize}

\paragraph{Express.js:}
\begin{itemize}
  \item \textbf{Ventajas:} Framework minimalista, rápido de implementar, ideal para APIs REST simples
  \item \textbf{Resultado:} Panel de administración funcional implementado en menos de una semana
\end{itemize}

\subsubsection{Metodología de trabajo}

El enfoque incremental por fases resultó efectivo:

\begin{itemize}
  \item \textbf{Fase 1 (consultas):} Estableció los fundamentos y patrones reutilizables
  \item \textbf{Fase 2 (creaciones):} Reutilizó componentes de Fase 1, acelerando el desarrollo
  \item \textbf{Fase 3 (campañas):} Integró aprendizajes de fases anteriores
  \item \textbf{Panel de administración:} Desarrollado en paralelo, no bloqueó automatización de HUs
\end{itemize}

Este enfoque permitió entregas parciales funcionales cada semana, facilitando revisión y ajustes tempranos.

\subsection{Recursos no utilizados pero considerados}

Durante la planificación se consideraron alternativas que finalmente no se implementaron:

\begin{itemize}
  \item \textbf{Docker:} Considerado para estandarizar el entorno de ejecución, pero descartado por la simplicidad del setup en Linux nativo y el overhead de recursos en un equipo con RAM limitada
  \item \textbf{CI/CD (GitHub Actions):} Planificado para integración continua, pero pospuesto por limitaciones de tiempo. Queda como recomendación futura
  \item \textbf{Base de datos:} Inicialmente considerada para almacenar datos de prueba, pero descartada en favor de estructuras de datos en memoria por simplicidad y velocidad
  \item \textbf{Framework de BDD (Cucumber):} Evaluado para especificaciones en Gherkin, pero descartado porque las especificaciones en TypeScript resultaron más mantenibles y no requirieron parser adicional
\end{itemize}
