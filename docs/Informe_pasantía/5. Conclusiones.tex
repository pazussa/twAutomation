\section{Conclusiones}
\label{sec:conclusiones}


Se desarrolló un sistema completo de gestión de datos de prueba implementado en el módulo \texttt{data.ts}, que incluye:
\begin{itemize}
  \item 29 intents con 942 variaciones de frases de entrada, proporcionando una cobertura exhaustiva de formulaciones de usuario
  \item Diccionario centralizado de variables contextuales (cultivos, fertilizantes, clientes, fechas) con funciones de materialización de plantillas
  \item Sistema de selección aleatoria de datos para evitar colisiones entre ejecuciones
  \item Funciones de mutación para reintentos automáticos ante conflictos de datos existentes
\end{itemize}

El sistema implementado elimina la dependencia de archivos externos, favoreciendo estructuras de datos en TypeScript que resultan más mantenibles y menos propensas a errores de sintaxis.


Se automatizaron las tres fases del proyecto con 30 especificaciones de prueba:
\begin{itemize}
  \item \textbf{Fase 1 (Consultas):} 15 HUs automatizadas incluyendo autenticación, listados y filtrados de recursos agrícolas
  \item \textbf{Fase 2 (Creación de productos):} 8 HUs automatizadas con flujos multi-turno de creación y asignación de precios
  \item \textbf{Fase 3 (Campañas y trabajos):} 7 HUs automatizadas con gestión de planificación agrícola completa
\end{itemize}

La arquitectura implementada basada en Page Object Model y sistema de reglas con expresiones regulares demostró ser altamente reutilizable. El desarrollo de componentes base en Fase 1 permitió acelerar significativamente la automatización de Fases 2 y 3.


Se implementó un sistema de reportes:
\begin{itemize}
  \item Generador de reportes personalizado (\texttt{conversation-reporter.ts}) con línea temporal detallada de cada mensaje enviado y recibido
  \item Exportación automática a PDF mediante script dedicado que utiliza Playwright para renderizado
  \item Nomenclatura con marca temporal para trazabilidad histórica
  \item Estadísticas globales con conteo de eventos exitosos, fallidos e intents ejecutados
\end{itemize}

Adicionalmente, se desarrolló un panel de administración web, que permite ejecución selectiva de pruebas y gestión de intents mediante interfaz gráfica.


\subsection{Conclusiones sobre aspectos técnicos}

La decisión de utilizar Playwright en lugar de Selenium o Puppeteer resultó crítica para el éxito del proyecto. Las ventajas observadas incluyen:
\begin{itemize}
  \item \textbf{Contexto persistente nativo:} Se eliminó completamente la necesidad de autenticación manual repetida con código QR, reduciendo el tiempo de setup de varios minutos a segundos
  \item \textbf{Esperas automáticas inteligentes:} Playwright maneja internamente muchas condiciones que requerirían esperas explícitas en Selenium
  \item \textbf{API moderna y tipado robusto:} Integración con TypeScript, reduciendo errores en tiempo de desarrollo
\end{itemize}

\paragraph{TypeScript sobre JavaScript}
El uso de TypeScript proporcionó beneficios tangibles:
\begin{itemize}
  \item Detección de errores en tiempo de compilación antes de ejecutar pruebas
  \item Autocompletado y documentación inline en el editor, acelerando el desarrollo
  \item Refactorizaciones seguras con verificación automática de impacto
  \item Interfaces y tipos personalizados que documentan la estructura de datos
\end{itemize}

Esta experiencia consolidó varios aprendizajes aplicables a futuros proyectos:

\begin{enumerate}
  \item \textbf{La flexibilidad es crucial en testing de sistemas no determinísticos:} Parsers rígidos y aserciones estrictas fallan ante la variabilidad natural de sistemas conversacionales
  
  \item \textbf{La inversión inicial en infraestructura se recupera rápidamente:} El tiempo dedicado a construir utilidades reutilizables en las primeras semanas aceleró significativamente el desarrollo posterior
  
  \item \textbf{La trazabilidad completa es esencial para debugging:} Los reportes detallados con timestamps y en cada mensaje intercambiado fueron muy valiosos para diagnosticar fallos
  
  \item \textbf{Las herramientas modernas reducen complejidad:} Playwright eliminó gran parte de la complejidad que caracterizaba a Selenium, permitiendo enfoque en lógica de negocio
  
\end{enumerate}

\subsubsection{Agradecimientos y reconocimientos}

Se agradece a Grandtek por proporcionar el contexto empresarial real para desarrollar esta pasantía, y a la Universidad del Cauca por la formación académica que hizo posible afrontar los desafíos técnicos con éxito.
