

\section{Actividades de la pasantía}
\label{sec:activities}

Durante la pasantía se ejecutó un conjunto de actividades orientadas a diseñar, implementar y operar un sistema de validación end-to-end de conversaciones de WhatsApp para un bot del dominio agrícola. Este trabajo implicó la construcción de un framework de automatización que incluye orquestación de pruebas, gestión de datos, ejecución automatizada de conversaciones y generación de reportes con evidencias detalladas.


\subsection{Descripción detallada de las actividades}

\subsubsection{Fase de preparación y configuración inicial}

La primera semana se dedicó a establecer los cimientos del proyecto de automatización. Se configuró un repositorio TypeScript con Playwright como motor de automatización de navegador.

\paragraph{Configuración del entorno de desarrollo}
Se estableció un proyecto Node.js con TypeScript, configurando compatibilidad ES2020 y gestión de módulos. El archivo de configuración del framework se parametrizó con timeouts extendidos (15 minutos por prueba) para permitir conversaciones multi-turno prolongadas.
\begin{figure}[h]
\centering
% PENDIENTE: Insertar captura de la estructura del proyecto
\includegraphics[width=0.8\textwidth]{figures/project-structure.png}
\caption{Estructura del proyecto de automatización}
\label{fig:project-structure}
\end{figure}

\paragraph{Manejo de sesión persistente en WhatsApp Web}
Un desafío crítico fue mantener la sesión de WhatsApp Web entre ejecuciones de pruebas sin requerir autenticación manual con código QR en cada inicio. Se implementó la estrategia de contexto persistente de Playwright, almacenando el estado del navegador (cookies, almacenamiento local y de sesión) en un directorio dedicado. El módulo de utilidades de autenticación contiene una función especializada que:

\begin{itemize}
  \item Verifica si existe una sesión activa comprobando elementos característicos de WhatsApp Web cargado
  \item Si no hay sesión, espera la presentación del código QR y la autenticación manual
  \item Una vez autenticado, persiste el estado para ejecuciones futuras
  \item Incluye mecanismos de reintento con esperas exponenciales para manejar latencias de red
\end{itemize}

\begin{figure}[h]
\centering
% PENDIENTE: Insertar captura del proceso de login
\includegraphics[width=0.7\textwidth]{figures/whatsapp-login.png}
\caption{Proceso de autenticación y persistencia de sesión }
\label{fig:whatsapp-login}
\end{figure}

\paragraph{Definición de convenciones y etiquetado}
Se estableció un sistema de nomenclatura y organización:

\begin{itemize}
  \item Archivos de prueba nombrados según la funcionalidad o historia de usuario que validan
  \item Estructura de carpetas organizada: directorio principal para especificaciones de prueba y subdirectorios para utilidades compartidas
  \item Etiquetas (tags) por fase y tipo de funcionalidad para permitir ejecución selectiva
  \item Convenciones de nombrado para variables de entorno en archivo de configuración
\end{itemize}

\subsubsection{Construcción de la base de automatización}

La segunda semana se enfocó en construir los componentes reutilizables que formarían el núcleo del framework de pruebas.

\paragraph{Page Objects y utilidades de interacción}
Siguiendo el patrón Page Object Model, se desarrolló un conjunto de funciones especializadas para interactuar con la interfaz de WhatsApp Web:

\begin{itemize}
  \item Función para localizar y abrir el chat con el contacto específico (el bot bajo prueba)
  \item Función para limpiar el historial del chat e iniciar con contexto limpio
  \item Función para escribir en el campo de texto del compositor de mensajes, manejando el foco correcto
  \item Función para ejecutar el envío del mensaje (mediante clic en botón o tecla Enter)
  \item Función para esperar activamente la aparición de una nueva burbuja de mensaje del bot
  \item Función para obtener solo los mensajes nuevos recibidos desde un punto de referencia
  \item Función para contar el total de burbujas de mensajes entrantes en el chat
\end{itemize}

Estas funciones encapsulan los selectores CSS y el manejo de condiciones de carrera propias de la interfaz web de mensajería.

\begin{figure}[h]
\centering
% PENDIENTE: Insertar diagrama del flujo de interacción
\includegraphics[width=0.9\textwidth]{figures/interaction-flow.png}
\caption{Flujo de interacción con WhatsApp Web mediante Page Objects }
\label{fig:interaction-flow}
\end{figure}

\paragraph{Esperas inteligentes y manejo de asincronía}
Dado que las respuestas del bot dependen de latencias de red, procesamiento del servidor y tiempos de entrega de WhatsApp, se implementaron estrategias de espera robustas:

\begin{itemize}
  \item \textbf{Esperas explícitas con polling:} En lugar de esperas fijas, se implementa un bucle que verifica periódicamente la aparición de nuevos mensajes, con tiempo límite configurable
  \item \textbf{Detección de patrones de texto:} Funciones especializadas que esperan hasta que aparezca un mensaje que coincida con un patrón específico
  \item \textbf{Reintentos acotados:} Se permite un número limitado de reintentos con incrementos exponenciales de tiempo cuando se detectan condiciones transitorias (ej. carga lenta de interfaz)
  \item \textbf{Tiempos límite configurables:} Los timeouts se parametrizan mediante variables de entorno para ajustar según las características del servidor de pruebas
\end{itemize}

\paragraph{Parsers de mensajes y lógica de decisión}
El bot responde con textos requeridos (listados, mensajes de existe o no existe, opciones). Se desarrollaron analizadores sintácticos (parsers) especializados:

\begin{itemize}
  \item Función para extraer el primer elemento de una lista y seleccionarlo automáticamente
  \item Sistema de reglas basado en expresiones regulares con prioridades, que analiza la respuesta del bot y determina la acción apropiada (responder con un valor específico, terminar exitosamente, terminar con error, etc.)
  \item Función que aplica las reglas sobre los mensajes recibidos y retorna la acción a ejecutar
  \item Función para resolver plantillas con variables del contexto actual (ej. reemplazar marcadores de posición con valores generados dinámicamente)
\end{itemize}

Este sistema permite que las pruebas sean declarativas y adapten su comportamiento dinámicamente según las respuestas del bot.

\begin{figure}[h]
\centering
% PENDIENTE: Insertar ejemplo de reglas y parseo
% \includegraphics[width=0.85\textwidth]{figures/keyword-rules.png}
\caption{Sistema de reglas por palabras clave para decisión automática}
\label{fig:keyword-rules}
\end{figure}

\paragraph{Modelo de datos y fixtures}
Se centralizó la gestión de datos de prueba en un módulo dedicado:

\begin{itemize}
  \item \textbf{Diccionario de variables:} Variables contextuales (nombres de cultivos, fertilizantes, clientes, fechas) que se materializan en las plantillas de mensajes
  \item \textbf{Mapa de intents:} Estructura que asocia cada intent (ej. ``crear cultivo'', ``listar fertilizantes'') con múltiples frases de ejemplo para probar variaciones de entrada
  \item \textbf{Funciones de selección aleatoria:} Funciones para seleccionar datos aleatorios y variar entre ejecuciones, evitando colisiones
  \item \textbf{Reinicio de variables:} Función para asegurar estado limpio al inicio de cada prueba
  \item \textbf{Mutaciones para reintentos:} Función que altera ligeramente los datos en caso de fallos transitorios para intentar nuevamente
\end{itemize}

\subsubsection{Automatización de Fase 1: Consultas básicas}

La segunda y tercera semana se automatizaron las historias de usuario de consulta y listado (Fase 1 del plan de trabajo):

\paragraph{HU-002: Autenticación y gestión de sesión}
Se implementaron escenarios para validar el flujo de solicitud de OTP por correo, ingreso de código válido, manejo de código inválido y detección de sesión expirada.

\paragraph{HU-006: Listado y filtrado de cultivos}
Las pruebas automatizadas implementan:

\begin{itemize}
  \item Listado completo de cultivos registrados
  \item Filtrado por nombre (coincidencia exacta y parcial)
  \item Manejo de casos sin resultados
  \item Creación exitosa de cultivos proporcionando todos los campos requeridos (nombre, variedad, marca, destino)
\end{itemize}

\begin{figure}[h]
\centering
% PENDIENTE: Insertar captura de ejecución de listado de cultivos
\includegraphics[width=0.9\textwidth]{figures/get-crops-execution.png}
\caption{Ejecución de listado y filtrado de cultivos}
\label{fig:get-crops-execution}
\end{figure}

\paragraph{HU-007 a HU-019: Consultas de recursos agrícolas}
Se automatizaron especificaciones individuales para:

\begin{itemize}
  \item \textbf{HU-008, HU-009, HU-010:} Listado de fertilizantes y fitosanitarios, con filtrado por nombre
  \item \textbf{HU-011, HU-012, HU-013:} Consultas de precios (último precio, variación histórica, precio mínimo)
  \item \textbf{HU-014:} Consulta de campos sin planificación
  \item \textbf{HU-017, HU-018, HU-019:} Consultas de distribución de cultivos, planificaciones de campaña y trabajos pendientes
\end{itemize}

Cada especificación de prueba sigue un patrón estructurado donde se define el caso de prueba y se ejecuta un bucle automático que:
\begin{enumerate}
  \item Limpia el chat para iniciar con contexto limpio
  \item Envía la frase de inicio del intent
  \item Entra en un bucle automático de lectura de respuestas y envío de acciones determinadas por las reglas definidas
  \item Registra cada paso en el sistema de trazabilidad conversacional
  \item Retorna resultado de éxito o fallo según los criterios de aceptación
\end{enumerate}

\subsubsection{Automatización de Fase 2: Creación y gestión de productos}

Durante la tercera semana se automatizaron los flujos de creación con validación exhaustiva de campos:

\paragraph{HU-020, HU-021, HU-022: Creación de productos}
Se implementaron especificaciones para crear cultivos, fertilizantes y productos químicos mediante flujos conversacionales completos:

\begin{itemize}
  \item Flujos de creación exitosa proporcionando todos los campos requeridos por el bot
  \item Interacción multi-turno donde el bot solicita secuencialmente cada campo necesario
  \item Verificación de que la creación se completa con mensaje de éxito del bot
\end{itemize}

Las especificaciones de prueba contienen múltiples variaciones de frases de inicio para probar diferentes formas de expresar el mismo intent.

\paragraph{HU-004: Asignación de precios}
Las pruebas implementan:

\begin{itemize}
  \item Flujo de asignación de precio a un producto mediante conversación con el bot
  \item Completar el proceso proporcionando producto, precio y fecha cuando el bot lo solicita
  \item Verificación de mensaje de confirmación exitosa del bot
\end{itemize}

\paragraph{HU-005: Búsqueda por materia activa}
Las pruebas verifican búsquedas exactas, parciales y casos sin coincidencias de productos fitosanitarios por su principio activo.

\paragraph{HU-003: Productos por fabricante}
Se valida el listado de productos de un fabricante específico, incluyendo casos con múltiples productos y fabricantes sin productos.

\subsubsection{Automatización de Fase 3: Campañas y trabajos}

En la cuarta semana se completaron los flujos más complejos relacionados con planificación agrícola:

\paragraph{HU-023: Creación de campañas}
Las pruebas implementan:

\begin{itemize}
  \item Flujo de creación de campaña proporcionando fecha de inicio
  \item Asociación de campaña con campos y cultivos mediante interacción conversacional
  \item Completar el proceso hasta recibir confirmación exitosa del bot
\end{itemize}

\paragraph{HU-024: Historial de campañas}
Las pruebas validan consultas de historial con:

\begin{itemize}
  \item Filtrado por rango de fechas
  \item Ordenamiento cronológico
  \item Detalles de cada campaña (cultivos asociados, trabajos planificados)
\end{itemize}

\paragraph{HU-025, HU-026, HU-027: Gestión de trabajos}
Se automatizaron:

\begin{itemize}
  \item \textbf{Planificación de trabajos}: asignación de tarea a un campo con fecha y tipo de labor
  \item \textbf{Reporte de trabajo finalizado}: confirmación de ejecución con cantidades y observaciones
  \item \textbf{Consultas de trabajos}: trabajos pendientes y último trabajo realizado
\end{itemize}

\begin{figure}[h]
\centering
% PENDIENTE: Insertar captura de flujo completo de campaña
\includegraphics[width=0.95\textwidth]{figures/campaign-flow.png}
\caption{Flujo end-to-end de creación y gestión de campaña}
\label{fig:campaign-flow}
\end{figure}

\subsubsection{Panel de administración y ejecución selectiva}

Paralelamente a la automatización de HUs, se desarrolló una interfaz web de administración para facilitar la ejecución de pruebas:

\paragraph{Servidor de administración}
Se implementó un servidor web local con Express y TypeScript que:

\begin{itemize}
  \item Sirve una interfaz web accesible localmente (puerto 3000)
  \item Lee dinámicamente todos los intents y ejemplos desde el módulo de datos, permitiendo editar variables y reglas de automatización.
  \item Expone servicios REST para obtener la lista de intents, ejemplos por intent, ejecutar selección de ejemplos y sirve como CRUD de intents y frases ejemplo.
  
\end{itemize}

\paragraph{Interfaz de usuario}
La interfaz web permite:

\begin{itemize}
  \item \textbf{Visualización agrupada:} Todos los intents listados con acordeón expandible, mostrando ejemplos dentro de cada intent
  \item \textbf{Ejecución con un clic:} Botón de ejecución que dispara la suite de pruebas seleccionadas mediante API
  \item \textbf{Edición de datos:} Formularios para agregar, editar o eliminar intents y ejemplos, persistiendo cambios en el repositorio de datos
  \item \textbf{Acceso directo a reportes:} Botón para abrir la carpeta de reportes generados
\end{itemize}

Este panel permite seleccionar, editar y ejecutar subconjuntos de pruebas mediante una interfaz gráfica sin necesidad de editar código.

\begin{figure}[h]
\centering
% PENDIENTE: Insertar captura del panel de administración
\includegraphics[width=0.95\textwidth]{figures/admin-panel.png}
\caption{Panel de administración para ejecución selectiva de pruebas}
\label{fig:admin-panel}
\end{figure}

\subsubsection{Sistema de reportes con evidencias detalladas}

Se desarrolló un generador de reportes personalizado que produce documentos HTML y PDF con:

\paragraph{Características del sistema de reportes}
\begin{itemize}
  \item \textbf{Línea temporal conversacional:} Cada mensaje enviado y recibido con marca de tiempo UTC precisa
  \item \textbf{Agrupación por intent:} Los reportes muestran cada intent ejecutado como una sección expandible
  \item \textbf{Indicadores visuales:} Insignias de estado (éxito/fallo) por intent y por paso individual
  \item \textbf{Estadísticas globales:} Resumen con total de eventos, exitosos, fallidos y número de intents ejecutados
  \item \textbf{Exportación automática a PDF:} Script que convierte los HTML a PDF preservando estilos, ejecutado automáticamente al finalizar cada suite
  \item \textbf{Nomenclatura con marca temporal:} Archivos nombrados con fecha y hora para trazabilidad histórica
\end{itemize}

\begin{figure}[h]
\centering
% PENDIENTE: Insertar ejemplo de reporte HTML
\includegraphics[width=0.9\textwidth]{figures/conversation-report.png}
\caption{Reporte de conversación con timeline y estados}
\label{fig:conversation-report}
\end{figure}

\paragraph{Estructura del reporte}
Cada reporte incluye:

\begin{verbatim}
[Título del test]
Status: passed / failed
Duración: X ms

Resumen:
- Eventos: N
- OK: X
- FAIL: Y
- Intents: Z

Conversación:
[1/Z] intent_name - OK
  >> Enviado: [mensaje]
  << Recibido: [respuesta bot]
  ...
[2/Z] otro_intent - FAIL
  >> Enviado: [mensaje]
  XX Error: [razón del fallo]
\end{verbatim}

Los reportes se almacenan en directorios organizados: formato HTML en un directorio de resultados de pruebas y formato PDF en un directorio de exportaciones.

\subsection{Relación con conocimientos académicos previos}

La ejecución del proyecto requirió la integración de conocimientos del programa de Ingeniería en Electrónica y Telecomunicaciones, énfasis Telemática:


\subsubsection{Programación y desarrollo de software}
\begin{itemize}
  \item Patrones de diseño: Page Object Model, Factory para generación de datos, Strategy para reglas de decisión
  \item Diseño de casos de prueba basados en criterios de aceptación
  \item Node.js: gestión de módulos, sistema de archivos, variables de entorno.
\end{itemize}



\subsection{Casos presentados, problemas y soluciones}

Durante la ejecución surgieron desafíos técnicos que requirieron análisis y soluciones creativas:

\subsubsection{Inestabilidad por latencia y sincronización}
\paragraph{Problema:} Las respuestas del bot tienen tiempos variables dependiendo de carga del servidor, complejidad de consulta y latencia de WhatsApp. Esperas fijas causaban fallos por timeout antes de recibir respuesta o tiempos de ejecución innecesariamente largos.

\paragraph{Solución implementada:}
\begin{itemize}
  \item Esperas explícitas con polling cada 200ms verificando aparición de nuevos mensajes mediante conteo de burbujas entrantes
  \item Timeout fijo de 45 segundos para todas las respuestas del bot, con esperas adicionales de 5 segundos para asegurar captura completa de mensajes
  \item Detección de patrones mediante sistema de reglas con expresiones regulares que analizan el contenido de los mensajes recibidos y determinan la acción a ejecutar
  \item Registro de timestamps UTC en reportes para análisis posterior de tiempos de respuesta
\end{itemize}

\subsubsection{Expiración de sesión de WhatsApp Web}
\paragraph{Problema:} WhatsApp Web expira sesiones después de períodos de inactividad o por reconexión desde otro dispositivo. Las pruebas fallaban al iniciar sin sesión válida.

\paragraph{Solución implementada:}
\begin{itemize}
  \item Uso de contexto persistente de Playwright (\texttt{launchPersistentContext}): almacena automáticamente las cookies, tokens de autenticación y estado de la sesión de WhatsApp Web en un directorio del sistema de archivos (\texttt{\textasciitilde/.wapp-autoloop-session}), evitando tener que escanear el código QR en cada ejecución
  \item Verificación al inicio: comprueba si el usuario ya está autenticado detectando elementos de la interfaz cargada; si no hay sesión activa, muestra el código QR y espera autenticación manual
  \item El directorio de sesión se reutiliza entre ejecuciones, manteniendo la sesión activa mientras WhatsApp no la expire por inactividad prolongada
\end{itemize}

\subsubsection{Confusión por acumulación de mensajes históricos}
\paragraph{Problema:} Después de múltiples ejecuciones de pruebas, el chat acumulaba cientos de mensajes. El sistema de conteo y extracción de mensajes se confundía al intentar identificar qué mensajes eran nuevos, causando que las pruebas procesaran respuestas antiguas como si fueran actuales o fallaran al no encontrar el contexto esperado.

\paragraph{Solución implementada:}
\begin{itemize}
  \item Implementación de función \texttt{clearChat()} que automatiza la limpieza del historial: localiza el botón de menú del chat, selecciona la opción ``Vaciar chat'' y confirma la acción
  \item Limpieza automática al inicio de cada prueba: cada conversación comienza con contexto limpio, eliminando mensajes anteriores
\end{itemize}

\subsubsection{Variaciones de formato en respuestas del bot}
\paragraph{Problema:} El bot alterna entre formatos de listados (numerados, con bullets, tabulados) y pequeñas variaciones de redacción. Selectores y aserciones rígidos fallaban con cambios menores.

\paragraph{Solución implementada:}
\begin{itemize}
  \item Analizadores basados en expresiones y palabras clave flexibles que detectan intención (ej. ``cualquier línea que contenga la palabra clave fertilizante'')
  \item Sistema de reglas con prioridades: reglas más específicas primero, genéricas como respaldo
  \item Extracción de opciones dinámica: funciones que identifican el formato y extraen el primer elemento. Entendiendo que los elementos están separados por comas.
  \item Tolerancia a espacios, mayúsculas/minúsculas y signos de puntuación
\end{itemize}





\subsubsection{Detección de bucles infinitos en conversaciones}
\paragraph{Problema:} Configuraciones incorrectas de reglas podían causar que el bot y la automatización entraran en bucle infinito (ej. bot pregunta → automación responde ``cancelar'' → bot pregunta de nuevo → loop).

\paragraph{Solución implementada:}
  Detección de patrones repetitivos: si se ve la misma pregunta del bot 3 veces consecutivas, abortar con error


\begin{figure}[h]
\centering
% PENDIENTE: Insertar captura de detección de bucle en reporte
\includegraphics[width=0.8\textwidth]{figures/loop-detection.png}
\caption{Detección de bucle infinito en reporte}
\label{fig:loop-detection}
\end{figure}


\subsection{Resultados y métricas}

Al finalizar la pasantía se alcanzaron las siguientes métricas:

\begin{itemize}
  \item \textbf{Cobertura:} 30 especificaciones de prueba automatizadas cubriendo las historias de usuario principales del sistema
  \item \textbf{Intenciones (intents):} 29 intents con múltiples frases de ejemplo cada una (total 942 variaciones de entrada)
  \item \textbf{Reportes generados:} Cada ejecución produce reporte HTML detallado con exportación automática a PDF mediante script dedicado
  \item \textbf{Panel de administración:} Interfaz web para gestión de intents, ejecución selectiva de pruebas y acceso a reportes
\end{itemize}

\subsection{Aprendizajes y recomendaciones}

\subsubsection{Aprendizajes clave}
\begin{itemize}
  \item La automatización de interfaces conversacionales requiere flexibilidad: parsers rígidos fallan ante cambios menores
  \item La trazabilidad completa (logs, timestamps, capturas) es esencial para depuración en entornos no determinísticos
  \item Un panel de administración visual facilita la ejecución de pruebas sin necesidad de conocimientos técnicos profundos
  \item Los datos de prueba deben ser versionados y aislados para reproducibilidad
\end{itemize}

\subsubsection{Recomendaciones futuras}
\begin{itemize}
  \item \textbf{Integración continua:} Integrar la suite en pipelines de integración continua para ejecución automática en cada cambio de código
  \item \textbf{Ejecución paralela:} Explorar uso de múltiples instancias con cuentas distintas para ejecutar pruebas en paralelo y reducir tiempo total
  \item \textbf{Integración con sistema de seguimiento:} Crear tickets automáticamente cuando las pruebas fallan de manera consistente
\end{itemize}


