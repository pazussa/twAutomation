\section{Anexos}
\label{sec:anexos}

\subsection{Anexo A: Repositorio del proyecto}

El código completo del proyecto de automatización de pruebas de WhatsApp se encuentra disponible en el repositorio público de GitHub:

\begin{center}
\textbf{Repositorio:} \url{https://github.com/pazussa/twAutomation}
\end{center}

\subsubsection{Contenido del repositorio}

El repositorio incluye:

\begin{itemize}
  \item \textbf{Código fuente completo:} Todas las especificaciones de prueba (\texttt{tests/*.spec.ts}), módulos de utilidades (\texttt{tests/setup/}), servidor de administración (\texttt{src/admin/}) y scripts auxiliares (\texttt{scripts/})
  
  \item \textbf{Configuración del proyecto:} Archivos \texttt{package.json}, \texttt{tsconfig.json}, \texttt{playwright.config.ts} con todas las dependencias y configuraciones necesarias
  
  \item \textbf{Documentación:} Archivo \texttt{README.md} con instrucciones detalladas de instalación, configuración y uso del sistema
  
  \item \textbf{Datos de prueba:} Módulo \texttt{tests/setup/data.ts} con los 29 intents y 942 variaciones de frases de entrada
  
  \item \textbf{Sistema de reportes:} Reportero personalizado (\texttt{tests/conversation-reporter.ts}) y script de exportación a PDF (\texttt{scripts/export-report-to-pdf.mjs})
  
  \item \textbf{Panel de administración:} Código del servidor Express y frontend HTML para gestión visual de intents
\end{itemize}

\subsubsection{Estructura del repositorio}

\begin{verbatim}
twAutomation/
  tests/
    *.spec.ts              (30 especificaciones de prueba)
    setup/
      data.ts              (Intents y datos de prueba)
      flow.ts              (Logica de flujos conversacionales)
      utils.ts             (Utilidades de interaccion WhatsApp)
    conversation-reporter.ts (Generador de reportes)
  src/
    admin/
      server.ts            (Servidor Express)
      public/              (Frontend del panel)
  scripts/
    export-report-to-pdf.mjs
    sync-yml-to-data.mjs
  package.json
  tsconfig.json
  playwright.config.ts
  README.md
\end{verbatim}

\subsubsection{Instrucciones de uso}

Para replicar el entorno de desarrollo y ejecutar las pruebas:

\begin{enumerate}
  \item Clonar el repositorio: \texttt{git clone https://github.com/pazussa/twAutomation}
  \item Instalar dependencias: \texttt{npm install}
  \item Configurar variables de entorno en archivo \texttt{.env}
  \item Ejecutar pruebas: \texttt{npx playwright test}
  \item Iniciar panel de administración: \texttt{npm run admin}
\end{enumerate}

Consultar el archivo \texttt{README.md} del repositorio para instrucciones detalladas de configuración y uso avanzado.

\subsection{Anexo B: Información de la empresa}

\subsubsection{Grandtek - Soluciones Especializadas de Software}

La pasantía se desarrolló en Grandtek, empresa especializada en desarrollo de soluciones de software basadas en tecnologías de la información y las comunicaciones (TIC).

\begin{center}
\textbf{Sitio web:} \url{https://grandtek.co/}
\end{center}

